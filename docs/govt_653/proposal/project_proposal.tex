\documentclass[]{article}
\usepackage{lmodern}
\usepackage{amssymb,amsmath}
\usepackage{ifxetex,ifluatex}
\usepackage{fixltx2e} % provides \textsubscript
\ifnum 0\ifxetex 1\fi\ifluatex 1\fi=0 % if pdftex
  \usepackage[T1]{fontenc}
  \usepackage[utf8]{inputenc}
\else % if luatex or xelatex
  \ifxetex
    \usepackage{mathspec}
  \else
    \usepackage{fontspec}
  \fi
  \defaultfontfeatures{Ligatures=TeX,Scale=MatchLowercase}
\fi
% use upquote if available, for straight quotes in verbatim environments
\IfFileExists{upquote.sty}{\usepackage{upquote}}{}
% use microtype if available
\IfFileExists{microtype.sty}{%
\usepackage{microtype}
\UseMicrotypeSet[protrusion]{basicmath} % disable protrusion for tt fonts
}{}
\usepackage[margin=1in]{geometry}
\usepackage{hyperref}
\hypersetup{unicode=true,
            pdftitle={Project Proposal},
            pdfauthor={Kiernan Nicholls},
            pdfborder={0 0 0},
            breaklinks=true}
\urlstyle{same}  % don't use monospace font for urls
\usepackage{longtable,booktabs}
\usepackage{graphicx,grffile}
\makeatletter
\def\maxwidth{\ifdim\Gin@nat@width>\linewidth\linewidth\else\Gin@nat@width\fi}
\def\maxheight{\ifdim\Gin@nat@height>\textheight\textheight\else\Gin@nat@height\fi}
\makeatother
% Scale images if necessary, so that they will not overflow the page
% margins by default, and it is still possible to overwrite the defaults
% using explicit options in \includegraphics[width, height, ...]{}
\setkeys{Gin}{width=\maxwidth,height=\maxheight,keepaspectratio}
\IfFileExists{parskip.sty}{%
\usepackage{parskip}
}{% else
\setlength{\parindent}{0pt}
\setlength{\parskip}{6pt plus 2pt minus 1pt}
}
\setlength{\emergencystretch}{3em}  % prevent overfull lines
\providecommand{\tightlist}{%
  \setlength{\itemsep}{0pt}\setlength{\parskip}{0pt}}
\setcounter{secnumdepth}{0}
% Redefines (sub)paragraphs to behave more like sections
\ifx\paragraph\undefined\else
\let\oldparagraph\paragraph
\renewcommand{\paragraph}[1]{\oldparagraph{#1}\mbox{}}
\fi
\ifx\subparagraph\undefined\else
\let\oldsubparagraph\subparagraph
\renewcommand{\subparagraph}[1]{\oldsubparagraph{#1}\mbox{}}
\fi

%%% Use protect on footnotes to avoid problems with footnotes in titles
\let\rmarkdownfootnote\footnote%
\def\footnote{\protect\rmarkdownfootnote}

%%% Change title format to be more compact
\usepackage{titling}

% Create subtitle command for use in maketitle
\newcommand{\subtitle}[1]{
  \posttitle{
    \begin{center}\large#1\end{center}
    }
}

\setlength{\droptitle}{-2em}

  \title{Project Proposal}
    \pretitle{\vspace{\droptitle}\centering\huge}
  \posttitle{\par}
  \subtitle{GOVT-653}
  \author{Kiernan Nicholls}
    \preauthor{\centering\large\emph}
  \postauthor{\par}
      \predate{\centering\large\emph}
  \postdate{\par}
    \date{March 7, 2019}


\begin{document}
\maketitle

I am hoping to further study the use of virtual prediction markets as
tools to predict American Congressional elections. Prediction markets
use self-interested traders betting on the outcome of an election to
produce a probabilistic view of potential outcomes.

I first began the process of this study last semester in GOVT-696
Programming Applied Political Data Science. During that class, I really
only began a rudimentary collection and analysis of the data. For this
class, I'm hoping to further explore the economic forces at work in such
a system, as well as the fundamentals of the alternative prediction
tools (polling and modeling).

\subsection{Literature Review}\label{literature-review}

Arrow, Kenneth J., et al. ``The Promise of Prediction Markets.''
Science, vol.~320, no. 5878, 2008, pp.~877--878. JSTOR,
www.jstor.org/stable/20054723. This article discusses the need for
further study on prediction markets in the context of the legal issues
surrounding the practice. The United States federal government has
legislation which outlaws many types of gambling, especially those
around elections. Arrow et al. suggest these laws ``create significant
barriers to the establishment of vibrant, liquid prediction markets in
the United States.'' The Unlawful Internet Gambling Act of 2006 hinders
the potential to utilize the internet as a perfect market to exchange
futures contracts. The article proposes two specific steps to facilitate
the use and study of prediction markets: (1) the Commodity Futures
Trading Commission (CFTC) should create safe-harbor rules for
small-stakes prediction markets, and (2) Congress should support
prediction markets with legislation to cover the increased spending at
the CTFC and secure a legal framework for prediction markets. The date
of this paper is noteworthy, as the CTFC does currently extend PredictIt
and the Iowa Election Market a ``no-action'' status due to their
academic value. This paper argues more action needs to be taken to truly
utilize and explore prediction markets.

Berg, Joyce E., and Thomas A. Rietz. ``Market Design, Manipulation, and
Accuracy in Political Prediction Markets: Lessons from the Iowa
Electronic Markets.'' PS: Political Science and Politics, vol.~47, no.
2, 2014, pp.~293--296. JSTOR, www.jstor.org/stable/43284536. This paper
explores the shortcomings of prediction markets, specifically the
potential for manipulation. The authors use the Iowa Election Market's
2012 Presidential election. The paper uses the Securities and Exchange
Act definition of price manipulation: ``To effect, alone or with 1 or
more other persons, a series of transactions\ldots{} raising or
depressing the price of (a) security, for the purpose of inducing the
purchase or sale of such security by others.'' The author's extend this
definition to political prediction markets by assuming manipulators are
using price manipulation not only to profit but as potential means to
change the outcome of the election itself. The paper does not weigh in
on the debate between ``bandwagon theory'' and ``underdog theory''
(which each try and explain the effect perception of victory has on
voter behaviour), only going as far as connecting each theory with
market price. The paper concludes that manipulation can affect election
prediction market forecasting accuracy, although the motives for
manipulators is unclear. Market design can mitigate the risk of
manipulation. They found little risk that the Iowa Election Market could
be successfully manipulated.

Fair, Ray C. ``Interpreting the Predictive Uncertainty of Elections.''
The Journal of Politics, vol.~71, no. 2, 2009, pp.~612--626. JSTOR,
www.jstor.org/stable/10.1017/s0022381609090495. Uncertainty is a
fundamental concept in the prediction of elections. This paper discusses
the nature of uncertainty in U.S. elections, the shortcoming of
``error'' in opinion pollings, and potential role for prediction markets
to help both social scientists the the population at large define and
understand this uncertainty. The theory of uncertainty assumes that on
election day there are n possible outcomes to an election, each with a
probability of 1/n. If in p percent of the n conditions a given
candidate wins, then then p is the probability of that candidate
winning. Fair explains that ``even if ne knew the n possible conditions
of nature, the best that one could say at the beginning of election day
is that A candidate would win with probability p'' (612). Weather is
given as one example cause of uncertainty on election day; given the
truly chaotic nature of weather, and it's unknown effect on voter
turnout, uncertainty must exist on election day. Prediction markets
offer insight into uncertainty due to the balance of interests and the
zero to one nature of the futures contracts being trades. Since the
outcomes are essentially limited to two (win or lose) and the date of
the event is known, the equilibrium price of the prediction market
should correlate with the probability of each outcome. Fair contrasts
this type of uncertainty with the standard error of opinion polling.
Fair explains that if you polled the entire population of voters the day
before, the poll would come with a standard error of zero despite
uncertainty still existing in nature (thanks to weather, for example).
The paper concludes with Fair's ``ranking theory'' which provides a role
for uncertainty in the strategic behaviour of political operatives in
their ranking of campaign priorities.

Lee, David S., and Enrico Moretti. ``Bayesian Learning and the Pricing
of New Information: Evidence from Prediction Markets.'' The American
Economic Review, vol.~99, no. 2, 2009, pp.~330--336. JSTOR,
www.jstor.org/stable/25592420.

This paper by David Lee and Enrico Moretti explored the topic of
prediction markets as they relate the Bayesian statistical theory. Since
prediction markets allow social scientists to explore the opinion of the
market over time in direct response to changes in information. The paper
explores the prediction market hosted on the now-defunct Intrade
regarding the 2008 Presidential Election. The paper explores the direct
relationship opinion polling has on prediction markets. In theory, the
market prices should reflect all information available to the traders,
with opinion polling being the most popular and useful in 2008. The
researchers found that ``rather than anticipate significant changes in
voter sentiment, the market price appears to be reacting to the release
of the polling information'' (330). In their model, investors have their
own opinion of each candidate's probability which is updated after
receiving the information obtained by a poll.

Northcott, Robert. ``Opinion Polling and Election Predictions.''
Philosophy of Science, vol.~82, no. 5, 2015, pp.~1260--1271. JSTOR,
www.jstor.org/stable/10.1086/683651.

In this paper, Robert Northcott says ``election prediction by means of
opinion polling is a rare empirical success story for social science''
(1260). In so much of social science, predicting future outcomes proves
difficult due to the open system nature of human elections. Northcott
cites many before him who have claimed there are simply too many
variables acting on too many unique individuals to establish a model.
Northcott takes a philosophical angle in his analysis of how opinion
polling has proven the exception to this understanding, providing social
scientists with a tool to predict the actions of a population in
advance. Northcott concludes that the success of opinion polling is no
mystery, but a result of scientifically rigorous methodology. The
accuracy of forecasting in the 2012 Presidential election by Huffington
Post, FiveThirtyEight, and Princeton Elections is, according to
Northcott, ``a stunning success story, arguably with few equals in
social science'' (1261). Northcott explores the two levels of opinion
polling, the first level demographic balancing and the potential second
level aggregation that has recently become popular. He addresses
sampling errors and election system idiosyncrasies as sources of errors
in polling's ability to translate to the true population. In this paper,
the difference between explaining an election and predicting an election
is discussed, an importance difference when expressing the need for
prediction in social science.

Silver, Nate. The Signal and the Noise: Why Most Predictions Fail but
Some Don't. Penguin Press, 2012.

In the preface to the print edition, Silver discusses the nature of
``big data'' and it's place in Gartner's hype cycle. When Silver's data
journalism venture FiveThirtyEight used a forecasting model to
accurately predicted all 50 states in the 2012 Presidential election,
big data potentially at the peak of the hype cycle. In the preface,
written in 2014, Silver worries data-based predictions might soon enter
the second phase of the hype cycle, the ``trough of disillusionment.''
This insight is particularly interesting in the wake of the 2016
Presidential election only two years later, where Silver's model was
potentially the most accurate of the prediction models, but still gave
the eventual loser of the election a probability of victory greater than
70\% on the eve of the election. While not a scholarly work, The Signal
and the Noise is useful in the context of this paper because
FiveThirtyEight provides the probabilistic prediction to which markets
would be compared. In the book, Silver explores the nature of
predictions and uncertainty, especially as they relate to elections.
Prediction markets are brought up throughout. In one instance; Silver
recalls being challenged to ``put his money where his mouth is'' (a key
concept in prediction market theory) when his model-based prediction for
the 2012 Republican Iowa Caucus contradicted the market price at the
time. Ultimately Silver won that bet, and he continues to challenge
reliance on prediction markets due to their fundamentally human nature.
The fallibility of human judgement is a key concept in this book. This
book provides crucial insight into Silver's FiveThirtyEight model and
the nature of probability and uncertainty in predictions.

\subsection{Theoretical Question}\label{theoretical-question}

Throughout the campaign, with varying levels of information available to
traders and models, can prediction markets offer a more accurate view of
probabilities compared to mathematical forecasting models? Data

I would use market data from PredictIt.com, which provides their data
for free for academic use. I have partned with the service to obtain
data on the 2018 midterm markets. Below is a random sample of market
data with selection of variables:

\begin{longtable}[]{@{}rlllrrrr@{}}
\caption{44,711 observations of 6 variables}\tabularnewline
\toprule
ID & Market & Contract & Date & Low & High & Close &
Volume\tabularnewline
\midrule
\endfirsthead
\toprule
ID & Market & Contract & Date & Low & High & Close &
Volume\tabularnewline
\midrule
\endhead
3883 & TX29.2018 & DEM.TX29.2018 & 2018-08-06 & 0.96 & 0.96 & 0.96 &
1\tabularnewline
3503 & KING.MESENATE.2018 & n/a & 2017-09-05 & 0.83 & 0.83 & 0.83 &
0\tabularnewline
3737 & WA08.2018 & DEM.WA08.2018 & 2018-02-07 & 0.77 & 0.81 & 0.77 &
3\tabularnewline
3489 & STAB.MISENATE.2018 & n/a & 2018-04-30 & 0.85 & 0.85 & 0.85 &
0\tabularnewline
3863 & TX05.2018 & GOP.TX05.2018 & 2018-01-20 & 0.86 & 0.86 & 0.86 &
0\tabularnewline
4263 & NC09.2018 & DEM.NC09.2018 & 2018-11-06 & 0.25 & 0.99 & 0.25 &
1243\tabularnewline
4016 & PA09.2018 & DEM.PA09.2018 & 2018-06-21 & 0.11 & 0.11 & 0.11 &
0\tabularnewline
2940 & SANDERS.VTSENATE.2018 & n/a & 2018-07-14 & 0.96 & 0.96 & 0.96 &
0\tabularnewline
3480 & HEIT.NDSENATE.2018 & n/a & 2017-09-07 & 0.56 & 0.60 & 0.60 &
30\tabularnewline
2941 & MANCHIN.WVSENATE.2018 & n/a & 2017-06-04 & 0.62 & 0.62 & 0.62 &
0\tabularnewline
\bottomrule
\end{longtable}

I would use model data from FiveThirtyEight, who provide top-level
outputs of their model for free to the public. Below is a selection of
variables from the 2018 House model:

\begin{longtable}[]{@{}llrllllrr@{}}
\caption{299,760 observations of 7 variables}\tabularnewline
\toprule
Date & State & District & Special & Party & Incumbent & Model & Win
Prob. & Avg Share\tabularnewline
\midrule
\endfirsthead
\toprule
Date & State & District & Special & Party & Incumbent & Model & Win
Prob. & Avg Share\tabularnewline
\midrule
\endhead
2018-09-02 & WV & 2 & NA & D & FALSE & lite & 0.087 &
41.43\tabularnewline
2018-08-06 & VA & 5 & NA & D & FALSE & deluxe & 0.411 &
49.12\tabularnewline
2018-09-29 & TN & 5 & NA & R & FALSE & deluxe & 0.000 &
32.33\tabularnewline
2018-10-31 & IL & 10 & NA & R & FALSE & lite & 0.000 &
27.13\tabularnewline
2018-08-15 & WY & 1 & NA & D & FALSE & deluxe & 0.000 &
29.28\tabularnewline
2018-10-30 & NY & 1 & NA & D & FALSE & classic & 0.071 &
43.70\tabularnewline
2018-08-31 & NY & 24 & NA & R & TRUE & deluxe & 0.875 &
53.76\tabularnewline
2018-10-19 & KY & 6 & NA & NA & FALSE & classic & 0.000 &
2.94\tabularnewline
2018-10-21 & CA & 43 & NA & D & TRUE & classic & 1.000 &
78.98\tabularnewline
2018-08-12 & OH & 6 & NA & D & FALSE & lite & 0.010 &
35.37\tabularnewline
\bottomrule
\end{longtable}

Ultimately, I would like to explore the accuracy of each tool in their
ability to predict elections. Furthermore, I would like to explore how
changes in each tool's inputs affect their accuracy (e.g., total trade
volume, market manipulation, opinion polling, Google search trends,
etc).

Preliminary R code to collect and format this data for comparison can be
found on my github repository for this project:
\url{https://github.com/kiernann/predictr}.


\end{document}
