\documentclass[]{article}
\usepackage{lmodern}
\usepackage{amssymb,amsmath}
\usepackage{ifxetex,ifluatex}
\usepackage{fixltx2e} % provides \textsubscript
\ifnum 0\ifxetex 1\fi\ifluatex 1\fi=0 % if pdftex
  \usepackage[T1]{fontenc}
  \usepackage[utf8]{inputenc}
\else % if luatex or xelatex
  \ifxetex
    \usepackage{mathspec}
  \else
    \usepackage{fontspec}
  \fi
  \defaultfontfeatures{Ligatures=TeX,Scale=MatchLowercase}
\fi
% use upquote if available, for straight quotes in verbatim environments
\IfFileExists{upquote.sty}{\usepackage{upquote}}{}
% use microtype if available
\IfFileExists{microtype.sty}{%
\usepackage{microtype}
\UseMicrotypeSet[protrusion]{basicmath} % disable protrusion for tt fonts
}{}
\usepackage[margin=1in]{geometry}
\usepackage{hyperref}
\hypersetup{unicode=true,
            pdftitle={Final Project Memo},
            pdfauthor={Kiernan Nicholls},
            pdfborder={0 0 0},
            breaklinks=true}
\urlstyle{same}  % don't use monospace font for urls
\usepackage{color}
\usepackage{fancyvrb}
\newcommand{\VerbBar}{|}
\newcommand{\VERB}{\Verb[commandchars=\\\{\}]}
\DefineVerbatimEnvironment{Highlighting}{Verbatim}{commandchars=\\\{\}}
% Add ',fontsize=\small' for more characters per line
\usepackage{framed}
\definecolor{shadecolor}{RGB}{248,248,248}
\newenvironment{Shaded}{\begin{snugshade}}{\end{snugshade}}
\newcommand{\KeywordTok}[1]{\textcolor[rgb]{0.13,0.29,0.53}{\textbf{#1}}}
\newcommand{\DataTypeTok}[1]{\textcolor[rgb]{0.13,0.29,0.53}{#1}}
\newcommand{\DecValTok}[1]{\textcolor[rgb]{0.00,0.00,0.81}{#1}}
\newcommand{\BaseNTok}[1]{\textcolor[rgb]{0.00,0.00,0.81}{#1}}
\newcommand{\FloatTok}[1]{\textcolor[rgb]{0.00,0.00,0.81}{#1}}
\newcommand{\ConstantTok}[1]{\textcolor[rgb]{0.00,0.00,0.00}{#1}}
\newcommand{\CharTok}[1]{\textcolor[rgb]{0.31,0.60,0.02}{#1}}
\newcommand{\SpecialCharTok}[1]{\textcolor[rgb]{0.00,0.00,0.00}{#1}}
\newcommand{\StringTok}[1]{\textcolor[rgb]{0.31,0.60,0.02}{#1}}
\newcommand{\VerbatimStringTok}[1]{\textcolor[rgb]{0.31,0.60,0.02}{#1}}
\newcommand{\SpecialStringTok}[1]{\textcolor[rgb]{0.31,0.60,0.02}{#1}}
\newcommand{\ImportTok}[1]{#1}
\newcommand{\CommentTok}[1]{\textcolor[rgb]{0.56,0.35,0.01}{\textit{#1}}}
\newcommand{\DocumentationTok}[1]{\textcolor[rgb]{0.56,0.35,0.01}{\textbf{\textit{#1}}}}
\newcommand{\AnnotationTok}[1]{\textcolor[rgb]{0.56,0.35,0.01}{\textbf{\textit{#1}}}}
\newcommand{\CommentVarTok}[1]{\textcolor[rgb]{0.56,0.35,0.01}{\textbf{\textit{#1}}}}
\newcommand{\OtherTok}[1]{\textcolor[rgb]{0.56,0.35,0.01}{#1}}
\newcommand{\FunctionTok}[1]{\textcolor[rgb]{0.00,0.00,0.00}{#1}}
\newcommand{\VariableTok}[1]{\textcolor[rgb]{0.00,0.00,0.00}{#1}}
\newcommand{\ControlFlowTok}[1]{\textcolor[rgb]{0.13,0.29,0.53}{\textbf{#1}}}
\newcommand{\OperatorTok}[1]{\textcolor[rgb]{0.81,0.36,0.00}{\textbf{#1}}}
\newcommand{\BuiltInTok}[1]{#1}
\newcommand{\ExtensionTok}[1]{#1}
\newcommand{\PreprocessorTok}[1]{\textcolor[rgb]{0.56,0.35,0.01}{\textit{#1}}}
\newcommand{\AttributeTok}[1]{\textcolor[rgb]{0.77,0.63,0.00}{#1}}
\newcommand{\RegionMarkerTok}[1]{#1}
\newcommand{\InformationTok}[1]{\textcolor[rgb]{0.56,0.35,0.01}{\textbf{\textit{#1}}}}
\newcommand{\WarningTok}[1]{\textcolor[rgb]{0.56,0.35,0.01}{\textbf{\textit{#1}}}}
\newcommand{\AlertTok}[1]{\textcolor[rgb]{0.94,0.16,0.16}{#1}}
\newcommand{\ErrorTok}[1]{\textcolor[rgb]{0.64,0.00,0.00}{\textbf{#1}}}
\newcommand{\NormalTok}[1]{#1}
\usepackage{graphicx,grffile}
\makeatletter
\def\maxwidth{\ifdim\Gin@nat@width>\linewidth\linewidth\else\Gin@nat@width\fi}
\def\maxheight{\ifdim\Gin@nat@height>\textheight\textheight\else\Gin@nat@height\fi}
\makeatother
% Scale images if necessary, so that they will not overflow the page
% margins by default, and it is still possible to overwrite the defaults
% using explicit options in \includegraphics[width, height, ...]{}
\setkeys{Gin}{width=\maxwidth,height=\maxheight,keepaspectratio}
\IfFileExists{parskip.sty}{%
\usepackage{parskip}
}{% else
\setlength{\parindent}{0pt}
\setlength{\parskip}{6pt plus 2pt minus 1pt}
}
\setlength{\emergencystretch}{3em}  % prevent overfull lines
\providecommand{\tightlist}{%
  \setlength{\itemsep}{0pt}\setlength{\parskip}{0pt}}
\setcounter{secnumdepth}{0}
% Redefines (sub)paragraphs to behave more like sections
\ifx\paragraph\undefined\else
\let\oldparagraph\paragraph
\renewcommand{\paragraph}[1]{\oldparagraph{#1}\mbox{}}
\fi
\ifx\subparagraph\undefined\else
\let\oldsubparagraph\subparagraph
\renewcommand{\subparagraph}[1]{\oldsubparagraph{#1}\mbox{}}
\fi

%%% Use protect on footnotes to avoid problems with footnotes in titles
\let\rmarkdownfootnote\footnote%
\def\footnote{\protect\rmarkdownfootnote}

%%% Change title format to be more compact
\usepackage{titling}

% Create subtitle command for use in maketitle
\newcommand{\subtitle}[1]{
  \posttitle{
    \begin{center}\large#1\end{center}
    }
}

\setlength{\droptitle}{-2em}

  \title{Final Project Memo}
    \pretitle{\vspace{\droptitle}\centering\huge}
  \posttitle{\par}
    \author{Kiernan Nicholls}
    \preauthor{\centering\large\emph}
  \postauthor{\par}
      \predate{\centering\large\emph}
  \postdate{\par}
    \date{October 16, 2018}


\begin{document}
\maketitle

\subsection{Proposal}\label{proposal}

I would like to study the the effectiveness of prediction markets as a
means of forecasting political outcomes. I hypothesize that prediction
markets serve as a more useful tool than any given public poll as a
means of predicting the outcome of elections in the United States.
Prediction markets theoretically encompass polling data in their
aggregation, in addition to the benefits derived from the wisdom of the
crowd and personal financial stakes.

\subsection{Prediction Markets}\label{prediction-markets}

Prediction markets are exchanges set up to trade ``stock'' in binary
outcomes. For any given event, traders can buy shares of each of the
binary outcomes: the event occurring or not. For every market,
``contracts'' are automatically established between two buyers. Traders
buy shares of each outcome based on their individual assessment of
likelihoods combined with a financial cost-benefit, risk-reward
analysis.

For example, a market might be established that asks the question ``Will
the Democratic party win a majority in the House of Representatives in
the 2018 Midterm Election?'' Trades can then buy ``Yes'' and ``No''
shares on that market. For every ``Yes'' buyer, there is a corresponding
``No'' buyer agreeing to a contract. The price of each share is a analog
for probability and must sum to 100. If the ``Yes'' buyer feels a share
is worth \$0.67, then the ``No'' buyer is agreeing that his shares are
worth \$0.33. When the event occurs in reality, the shares of each
outcome are automatically sold for \$1 or \$0 depending on the outcome.
If the Democratic party does win a majority, the ``Yes'' shares become
worth \$1 each and the ``No'' shares become worthless. Essentially, the
winner of the bet receives the funds wagered by his counterpart.

Theoretically, we can use these markets as a predictive tool. In
aggregate, the average trading price for each binary outcome should
reflect the probability of that outcome. Both the ``Yes'' and ``No''
buyers will want to maximize their financial gain and minimize their
loss. Traders will not buy shares at a price higher or their believed
worth. You would not buy a ``Yes'' share of an event at \$0.67 if you
believe that event is unlikely to happen. In real time, the average
trading price of shares reflect the aggregate belief of the market.
Traders will theoretically take into account all available information
to make smart trades, including traditional polling. Prediction markets
do away with the mathematical nuances of random sampling and polling
weights.

As of right now, shares of Democrats winning a majority in the midterms
are trading for \$0.67 PredictIt.org market. This can be interpreted as
a 66\% probability. Meanwhile, the forecasting method used by
FiveThirtyEight.com places that same probability at 84.8\%. Traders on
the prediction market are less confident than the mathematical model. If
my hypothesis is true, prediction markets should correctly predict the
outcome of elections more often than polling or modeling.

\subsection{PredictIt}\label{predictit}

PredictIt.org is a prediction market run by non-profit Victoria
University of Wellington, New Zealand for educational purposes. Users of
this site can legally trade on the exchange with real currency, winning
real money if their prediction comes true and their shares sell for \$1.
Traders can also sell their shares at any time, provided there is a
corresponding buyer; as public opinion shifts, ``Yes'' or ``No'' shares
may become more or less valuable. The site is exempt from the usual ban
on both online and political betting by working with researchers to
study prediction markets as a political tool. In October of 2014,
Commodity Futures Trading Commission granted the site a No Action letter
allowing them to operate.

The site hosts markets on nearly every conceivable political event,
electoral or otherwise. Will Donald Trump be president at year-end 2018?
Will the federal government be shut down on February 9? Will Ted Cruz be
re-elected to the U.S. Senate in Texas in 2018? Will Facebook's Mark
Zuckerberg run for president in 2020? How many tweets will
@realDonaldTrump post from noon Oct. 10 to noon Oct. 17?

I am interested in the election markets, comparing them to both
individual polling (Sienna College Poll), polling averages
(\url{https://realclearpolitics.com}), and forecast modeling
(\url{https://fivethirtyeight.com}) as a means of predicting the winner
of political elections.

\subsection{Example Data}\label{example-data}

\begin{Shaded}
\begin{Highlighting}[]
\KeywordTok{library}\NormalTok{(tidyverse)}
\KeywordTok{library}\NormalTok{(gtrendsR)}
\KeywordTok{library}\NormalTok{(lubridate)}
\end{Highlighting}
\end{Shaded}

First, we are going to read in FiveThirtyEight's 2018 Midterm forecast
data for the House of Representatives. This is a proprietary model used
to predict the outcome of the election. The model takes into account
adjusted polling, their CANTOR method (polling from nearby districts),
and a set of fundamental factors (fundraising, incumbency). Ultimately,
we are looking for their predicted probability of the Democratic party
winning control of the House of Representatives.

\begin{Shaded}
\begin{Highlighting}[]
\NormalTok{forecast <-}\StringTok{ }\KeywordTok{read_csv}\NormalTok{(}\StringTok{"https://goo.gl/E9XJUf"}\NormalTok{)}
\NormalTok{forecast <-}\StringTok{ }\KeywordTok{filter}\NormalTok{(forecast, party }\OperatorTok{==}\StringTok{ "D"}\NormalTok{)}
\NormalTok{forecast <-}\StringTok{ }\KeywordTok{select}\NormalTok{(forecast, forecastdate, win_probability)}
\KeywordTok{colnames}\NormalTok{(forecast) <-}\StringTok{ }\KeywordTok{c}\NormalTok{(}\StringTok{"date"}\NormalTok{, }\StringTok{"forecast.prob"}\NormalTok{)}
\end{Highlighting}
\end{Shaded}

Then, we are going to do the same for PredictIt's public market data.
Academic researchers have access to more detailed data, but this will do
for now. Similarly, we are only interested in this market's probability
of the Democratic party winning control of the House of Representatives.
We will use the closing price of the ``Democratic'' share for every day
as an analog for probability.

\begin{Shaded}
\begin{Highlighting}[]
\NormalTok{market <-}\StringTok{ }\KeywordTok{read_csv}\NormalTok{(}\StringTok{"https://goo.gl/w4K8pP"}\NormalTok{)}
\NormalTok{market <-}\StringTok{ }\KeywordTok{filter}\NormalTok{(market, ContractName }\OperatorTok{==}\StringTok{ "Democratic"}\NormalTok{)}
\NormalTok{market <-}\StringTok{ }\KeywordTok{select}\NormalTok{(market, DateString, CloseSharePrice)}
\KeywordTok{colnames}\NormalTok{(market) <-}\StringTok{ }\KeywordTok{c}\NormalTok{(}\StringTok{"date"}\NormalTok{, }\StringTok{"market.price"}\NormalTok{)}
\end{Highlighting}
\end{Shaded}

Finally, we can merge these two data frames by a shared date to see how
the market and forecast change over time.

\begin{Shaded}
\begin{Highlighting}[]
\NormalTok{predict <-}\StringTok{ }\KeywordTok{as_tibble}\NormalTok{(}\KeywordTok{merge}\NormalTok{(forecast, market))}
\NormalTok{predict <-}\StringTok{ }\NormalTok{predict }\OperatorTok\StringTok{ }
\StringTok{  }\KeywordTok{gather}\NormalTok{(}\StringTok{"market.price"}\NormalTok{, }
         \StringTok{"forecast.prob"}\NormalTok{, }
         \DataTypeTok{key =} \StringTok{"tool"}\NormalTok{,}
         \DataTypeTok{value =} \StringTok{"prob"}\NormalTok{) }\OperatorTok\StringTok{ }
\StringTok{  }\KeywordTok{arrange}\NormalTok{(date)}

\KeywordTok{ggplot}\NormalTok{(}\DataTypeTok{data =}\NormalTok{ predict) }\OperatorTok{+}
\StringTok{  }\KeywordTok{geom_line}\NormalTok{(}\DataTypeTok{mapping =} \KeywordTok{aes}\NormalTok{(}\DataTypeTok{x =}\NormalTok{ date, }
                          \DataTypeTok{y =}\NormalTok{ prob, }
                          \DataTypeTok{color =}\NormalTok{ tool)) }\OperatorTok{+}\StringTok{ }
\StringTok{  }\KeywordTok{coord_cartesian}\NormalTok{(}\DataTypeTok{ylim =} \KeywordTok{c}\NormalTok{(}\DecValTok{0}\NormalTok{, }\DecValTok{1}\NormalTok{)) }\OperatorTok{+}\StringTok{ }
\StringTok{  }\KeywordTok{labs}\NormalTok{(}\DataTypeTok{title =} \StringTok{"Market and Forceast Probabilites of Democratic Midterm Victory"}\NormalTok{,}
       \DataTypeTok{x =} \StringTok{"Date"}\NormalTok{,}
       \DataTypeTok{y =} \StringTok{"Probability / Market Price"}\NormalTok{,}
       \DataTypeTok{color =} \StringTok{"Predictive Tool"}\NormalTok{)}
\end{Highlighting}
\end{Shaded}

\includegraphics{nicholls_memo_files/figure-latex/plot-1.pdf}

Here we can see the market price continually underestimate the
Democrat's chances compared to the forecasting model used by
FiveThirtyEight. Still, the trends between the two predictions are
fairly closely linked; when one drops, so does the other, and vice
versa.

\subsection{Further Questions}\label{further-questions}

I would be interested in seeing if we can statistically establishing if
the FiveThirtyEight forecast or RealClearPolitics polling \emph{causes}
changes in market price, or if they move independently.

Once I partner with PredictIt.org as a research partner, I can gain
access to their closed markets. A powerful test in predictive
capabilities would be to compare the success rate of the three
predictive tools (market, model, poll) in, say, the 2014 Midterms or the
2016 Election.

I would also be interested in using Google Trends data to see how the
news affects the prediction market (and how Google searches affect both
polling and market price).

Here I am plotting the Google Trends ``hits'' value for ``Bill Nelson''
against the market price for Senator Nelson's election on PredictIt.

\begin{Shaded}
\begin{Highlighting}[]
\NormalTok{nelson.trend <-}\StringTok{ }\KeywordTok{gtrends}\NormalTok{(}\DataTypeTok{keyword =} \StringTok{"Bill Nelson"}\NormalTok{,}
                  \DataTypeTok{time =} \StringTok{"today 3-m"}\NormalTok{,}
                  \DataTypeTok{gprop =} \StringTok{"web"}\NormalTok{,}
                  \DataTypeTok{geo =} \StringTok{"US"}\NormalTok{)}
\NormalTok{nelson.trend <-}\StringTok{ }\NormalTok{nelson.trend}\OperatorTok{$}\NormalTok{interest_over_time}
\NormalTok{nelson.trend <-}\StringTok{ }\KeywordTok{as_tibble}\NormalTok{(}\KeywordTok{select}\NormalTok{(nelson.trend, date, hits))}
\NormalTok{nelson.trend}\OperatorTok{$}\NormalTok{date <-}\StringTok{ }\NormalTok{lubridate}\OperatorTok{::}\KeywordTok{as_date}\NormalTok{(nelson.trend}\OperatorTok{$}\NormalTok{date)}

\NormalTok{nelson.market <-}\StringTok{ }\KeywordTok{read_csv}\NormalTok{(}\StringTok{"https://www.predictit.org/Resource/DownloadMarketChartData?marketid=2999&timespan=90D"}\NormalTok{)}
\NormalTok{nelson.market <-}\StringTok{ }\KeywordTok{select}\NormalTok{(nelson.market, DateString, CloseSharePrice)}
\KeywordTok{colnames}\NormalTok{(nelson.market) <-}\StringTok{ }\KeywordTok{c}\NormalTok{(}\StringTok{"date"}\NormalTok{, }\StringTok{"market.price"}\NormalTok{)}

\NormalTok{nelson <-}\StringTok{ }\KeywordTok{as_tibble}\NormalTok{(}\KeywordTok{merge}\NormalTok{(nelson.trend, nelson.market))}
\NormalTok{nelson}\OperatorTok{$}\NormalTok{market.price <-}\StringTok{ }\NormalTok{nelson}\OperatorTok{$}\NormalTok{market.price }\OperatorTok{*}\StringTok{ }\DecValTok{100}

\NormalTok{nelson <-}\StringTok{ }\NormalTok{nelson }\OperatorTok\StringTok{ }
\StringTok{  }\KeywordTok{gather}\NormalTok{(}\StringTok{"hits"}\NormalTok{, }\StringTok{"market.price"}\NormalTok{, }
         \DataTypeTok{key =} \StringTok{"tool"}\NormalTok{,}
         \DataTypeTok{value =} \StringTok{"value"}\NormalTok{) }\OperatorTok\StringTok{ }
\StringTok{  }\KeywordTok{arrange}\NormalTok{(date)}

\KeywordTok{ggplot}\NormalTok{(}\DataTypeTok{data =}\NormalTok{ nelson) }\OperatorTok{+}
\StringTok{  }\KeywordTok{geom_line}\NormalTok{(}\DataTypeTok{mapping =} \KeywordTok{aes}\NormalTok{(}\DataTypeTok{x =}\NormalTok{ date, }\DataTypeTok{y =}\NormalTok{ value, }\DataTypeTok{color =}\NormalTok{ tool)) }\OperatorTok{+}
\StringTok{  }\KeywordTok{labs}\NormalTok{(}\DataTypeTok{title =} \StringTok{"Google Trends Hits and Prediction Market Prices"}\NormalTok{,}
       \DataTypeTok{x =} \StringTok{"Date"}\NormalTok{,}
       \DataTypeTok{y =} \StringTok{"Hits / Adjusted Market Price"}\NormalTok{,}
       \DataTypeTok{color =} \StringTok{"Tool"}\NormalTok{)}
\end{Highlighting}
\end{Shaded}

\includegraphics{nicholls_memo_files/figure-latex/nelson-1.pdf}

We can see that the two spikes in search volume match with a drop in his
market price (him doing poorly) and a rise in his market price (doing
well).

\subsection{Conclusion}\label{conclusion}

I think there is serious value in prediction markets as a tool in
political science. Polling can inaccurately capture a sample or
incorrectly weigh the responses. Forecasting relies on complex and
proprietary modeling, necessitating trust in the modelers.


\end{document}
