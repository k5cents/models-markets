% Options for packages loaded elsewhere
\PassOptionsToPackage{unicode=true}{hyperref}
\PassOptionsToPackage{hyphens}{url}
%
\documentclass[
  ignorenonframetext,
]{beamer}
\usepackage{pgfpages}
\setbeamertemplate{caption}[numbered]
\setbeamertemplate{caption label separator}{: }
\setbeamercolor{caption name}{fg=normal text.fg}
\beamertemplatenavigationsymbolsempty
% Prevent slide breaks in the middle of a paragraph
\widowpenalties 1 10000
\raggedbottom
\setbeamertemplate{part page}{
  \centering
  \begin{beamercolorbox}[sep=16pt,center]{part title}
    \usebeamerfont{part title}\insertpart\par
  \end{beamercolorbox}
}
\setbeamertemplate{section page}{
  \centering
  \begin{beamercolorbox}[sep=12pt,center]{part title}
    \usebeamerfont{section title}\insertsection\par
  \end{beamercolorbox}
}
\setbeamertemplate{subsection page}{
  \centering
  \begin{beamercolorbox}[sep=8pt,center]{part title}
    \usebeamerfont{subsection title}\insertsubsection\par
  \end{beamercolorbox}
}
\AtBeginPart{
  \frame{\partpage}
}
\AtBeginSection{
  \ifbibliography
  \else
    \frame{\sectionpage}
  \fi
}
\AtBeginSubsection{
  \frame{\subsectionpage}
}
\usepackage{lmodern}
\usepackage{amssymb,amsmath}
\usepackage{ifxetex,ifluatex}
\ifnum 0\ifxetex 1\fi\ifluatex 1\fi=0 % if pdftex
  \usepackage[T1]{fontenc}
  \usepackage[utf8]{inputenc}
  \usepackage{textcomp} % provides euro and other symbols
\else % if luatex or xelatex
  \usepackage{unicode-math}
  \defaultfontfeatures{Scale=MatchLowercase}
  \defaultfontfeatures[\rmfamily]{Ligatures=TeX,Scale=1}
\fi
% Use upquote if available, for straight quotes in verbatim environments
\IfFileExists{upquote.sty}{\usepackage{upquote}}{}
\IfFileExists{microtype.sty}{% use microtype if available
  \usepackage[]{microtype}
  \UseMicrotypeSet[protrusion]{basicmath} % disable protrusion for tt fonts
}{}
\makeatletter
\@ifundefined{KOMAClassName}{% if non-KOMA class
  \IfFileExists{parskip.sty}{%
    \usepackage{parskip}
  }{% else
    \setlength{\parindent}{0pt}
    \setlength{\parskip}{6pt plus 2pt minus 1pt}}
}{% if KOMA class
  \KOMAoptions{parskip=half}}
\makeatother
\usepackage{xcolor}
\IfFileExists{xurl.sty}{\usepackage{xurl}}{} % add URL line breaks if available
\IfFileExists{bookmark.sty}{\usepackage{bookmark}}{\usepackage{hyperref}}
\hypersetup{
  pdftitle={Markets and Models},
  pdfauthor={Kiernan Nicholls},
  hidelinks,
}
\urlstyle{same} % disable monospaced font for URLs
\newif\ifbibliography
\usepackage{graphicx,grffile}
\makeatletter
\def\maxwidth{\ifdim\Gin@nat@width>\linewidth\linewidth\else\Gin@nat@width\fi}
\def\maxheight{\ifdim\Gin@nat@height>\textheight\textheight\else\Gin@nat@height\fi}
\makeatother
% Scale images if necessary, so that they will not overflow the page
% margins by default, and it is still possible to overwrite the defaults
% using explicit options in \includegraphics[width, height, ...]{}
\setkeys{Gin}{width=\maxwidth,height=\maxheight,keepaspectratio}
\setlength{\emergencystretch}{3em} % prevent overfull lines
\providecommand{\tightlist}{%
  \setlength{\itemsep}{0pt}\setlength{\parskip}{0pt}}
\setcounter{secnumdepth}{-\maxdimen} % remove section numbering

% Set default figure placement to htbp
\makeatletter
\def\fps@figure{htbp}
\makeatother


\title{Markets and Models}
\author{Kiernan Nicholls}
\date{Spring, 2019}

\begin{document}
\frame{\titlepage}

\begin{frame}{Why predict elections?}
\protect\hypertarget{why-predict-elections}{}

\begin{itemize}
\tightlist
\item
  Resource allocation
\item
  Strategy adjustment
\item
  Quantitative journalism
\item
  Uncertainty is scary
\end{itemize}

\end{frame}

\begin{frame}{How to Predict Elections}
\protect\hypertarget{how-to-predict-elections}{}

\begin{enumerate}
\tightlist
\item
  Opinion Polls
\item
  Poll Aggregation
\item
  Forecast Models
\item
  Prediction Markets
\end{enumerate}

\end{frame}

\begin{frame}{Onion Polling}
\protect\hypertarget{onion-polling}{}

Ex: \emph{Washington Post/ABC}

\begin{itemize}
\tightlist
\item
  Simple random sampling
\item
  Response rates
\item
  Sample size
\item
  Statistical bias
\item
  Partisanship
\end{itemize}

In 1824, \emph{The Harrisburg Pennsylvanian} had Jackson over Adams, 335
to 169.

\emph{Literary Digest} starting polling nationally in 1916. Infamously
``sampled'' 2.3 million readers in 1936 and bias caused them to predict
Landon over over Roosevelt.

\end{frame}

\begin{frame}{Polling Aggregation}
\protect\hypertarget{polling-aggregation}{}

Ex: \emph{RealClearPolitics.com}

\begin{itemize}
\tightlist
\item
  21st century invention
\item
  Average out all polls
\item
  Law of large numbers
\item
  Minimize errors and reduce bias
\end{itemize}

\end{frame}

\begin{frame}{Forecasting Models}
\protect\hypertarget{forecasting-models}{}

\begin{quote}
(Forecasting models) take lots of polls, perform various types of
adjustments to them, and then blend them with other kinds of empirically
useful indicators\ldots{} to forecast each race. Then they account for
the uncertainty in the forecast and simulate the election thousands of
times.
\end{quote}

\end{frame}

\begin{frame}{Forecasting Models}
\protect\hypertarget{forecasting-models-1}{}

\begin{quote}
Most election models work in something like the following way: First,
they calculate the most likely outcome in a particular state (``The
Republican wins by 1 point'') and then they determine the degree of
uncertainty around that estimate. Most models do this by means of a
normal distribution or something similar to it.
\end{quote}

Historic indicators of \emph{greater} uncertainty:

\begin{enumerate}
\tightlist
\item
  The election is further away in time
\item
  There are fewer polls
\item
  Those polls disagree more with one another
\item
  The polling average disagrees more with the state fundamentals
\item
  There are more undecideds or third-party voters in the polls
\item
  The race is more lopsided
\end{enumerate}

\end{frame}

\begin{frame}{Model Inputs}
\protect\hypertarget{model-inputs}{}

\begin{enumerate}
\tightlist
\item
  \textbf{Polling:} District level polling, weighted for historical
  accuracy.
\item
  \textbf{CANTOR:} A proprietary algorithm to identify similar districts
  to infers results for polling-sparce districts.
\item
  \textbf{Expert forecasts:} Ratings published by the historically
  accurate experts.
\item
  \textbf{Fundamentals:} Non-polling factors, historically useful
  factors:

  \begin{itemize}
  \tightlist
  \item
    Incumbency
  \item
    Partisanship
  \item
    Previous margin
  \item
    Generic ballot
  \item
    Fundraising
  \item
    Scandals
  \end{itemize}
\end{enumerate}

\end{frame}

\begin{frame}{Model Outputs}
\protect\hypertarget{model-outputs}{}

\begin{table}

\caption{\label{tab:model_data}Model Data (299,760 observations with 7 of 12 variables)}
\centering
\begin{tabular}[t]{l|l|r|l|l|r|r}
\hline
Date & State & District & Party & Incumbent & Prob & Share\\
\hline
2018-08-01 & AK & 1 & R & TRUE & 0.72 & 49.35\\
\hline
2018-08-01 & AK & 1 & D & FALSE & 0.28 & 44.11\\
\hline
2018-08-01 & AL & 1 & R & TRUE & 1.00 & 64.90\\
\hline
2018-08-01 & AL & 1 & D & FALSE & 0.00 & 35.10\\
\hline
2018-08-01 & AL & 2 & R & TRUE & 0.97 & 58.23\\
\hline
2018-08-01 & AL & 2 & D & FALSE & 0.03 & 41.77\\
\hline
2018-08-01 & AL & 3 & R & TRUE & 1.00 & 62.27\\
\hline
2018-08-01 & AL & 3 & D & FALSE & 0.00 & 37.73\\
\hline
2018-08-01 & AL & 4 & R & TRUE & 1.00 & 76.32\\
\hline
2018-08-01 & AL & 4 & D & FALSE & 0.00 & 23.68\\
\hline
\end{tabular}
\end{table}

\end{frame}

\begin{frame}{Prediction Markets}
\protect\hypertarget{prediction-markets}{}

\begin{itemize}
\tightlist
\item
  Exchange-traded binary options markets
\item
  Contract price reflects probability
\item
  Crowd-sourcing information
\item
  Efficient market hypothesis
\item
  Price discovery through equilibrium
\item
  Risk aversion overcomes bias
\item
  Dubious legality in the United States
\end{itemize}

In 1503, traders bet on Papal successor.

Iowa Election Market founded in 1988.

\end{frame}

\begin{frame}{PredictIt}
\protect\hypertarget{predictit}{}

\begin{quote}
PredictIt is a unique and exciting real money site that tests your
knowledge of political events by letting you trade shares on everything
from the outcome of an election to a Supreme Court decision to major
world events\ldots{} PredictIt is run by Victoria University of
Wellington, New Zealand, a not-for-profit university, for educational
purposes
\end{quote}

\end{frame}

\begin{frame}{PredictIt Contracts}
\protect\hypertarget{predictit-contracts}{}

\begin{itemize}
\tightlist
\item
  Real money, \$850 limit imposed by CFTC
\item
  Elections, Justice, Administration, World
\item
  Futures contracts, executes at time or condition
\item
  Two buyers on either side
\item
  Execute for \$1 or \$0 based on outcome
\item
  Traders can sell at any time, price change reflects information
\end{itemize}

\end{frame}

\begin{frame}{PredictIt Markets}
\protect\hypertarget{predictit-markets}{}

\begin{itemize}
\tightlist
\item
  Who will win the 2020 Democratic presidential nomination?
\item
  Will Donald Trump be impeached in his first term?
\item
  Will Congress ratify the USMCA by year-end 2019?
\item
  Will Facebook's Mark Zuckerberg run for president in 2020?
\item
  How many tweets will @realDonaldTrump post from noon Mar.~22 to noon
  Mar.~29?
\item
  Will Theresa May be prime minister of the United Kingdom on 6/30?
\end{itemize}

\end{frame}

\begin{frame}{PredicIt Data}
\protect\hypertarget{predicit-data}{}

\begin{table}

\caption{\label{tab:market_data}Market Data (44,711 observations with 6 of 11 variables)}
\centering
\begin{tabular}[t]{l|l|l|r|r}
\hline
ID & Ticker & Date & Price & Volume\\
\hline
2918 & WARREN.MASENATE.2018 & 2017-09-23 & 0.85 & 9\\
\hline
3857 & FEIN.AZSENATE.2018 & 2018-04-04 & 0.79 & 292\\
\hline
4255 & MN03.2018 & 2018-04-14 & 0.36 & 1\\
\hline
3496 & MENE.NJSENATE.2018 & 2018-04-27 & 0.84 & 0\\
\hline
3812 & AZSEN18 & 2018-05-13 & 0.69 & 0\\
\hline
2928 & CRUZ.TXSENATE.2018 & 2018-07-06 & 0.79 & 384\\
\hline
3883 & TX29.2018 & 2018-07-23 & 0.96 & 0\\
\hline
3738 & FL27.2018 & 2018-08-09 & 0.94 & 0\\
\hline
3857 & FEIN.AZSENATE.2018 & 2018-08-20 & 0.93 & 27\\
\hline
4315 & FL15.2018 & 2018-09-28 & 0.70 & 0\\
\hline
\end{tabular}
\end{table}

\end{frame}

\begin{frame}{Messy Data}
\protect\hypertarget{messy-data}{}

\begin{table}

\caption{\label{tab:unnamed-chunk-1}Messy Combined (9,200 observations of 4 variables)}
\centering
\begin{tabular}[t]{l|l|r|r}
\hline
Date & Race & Market Price & Model Probability\\
\hline
2018-08-01 & AZ-S1 & 0.66 & 0.738\\
\hline
2018-08-01 & CA-12 & 0.91 & 1.000\\
\hline
2018-08-01 & CA-22 & 0.30 & 0.049\\
\hline
2018-08-01 & CA-25 & 0.61 & 0.745\\
\hline
2018-08-01 & CA-39 & 0.61 & 0.377\\
\hline
2018-08-01 & CA-48 & 0.72 & 0.666\\
\hline
2018-08-01 & CA-49 & 0.74 & 0.795\\
\hline
2018-08-01 & CA-S1 & 0.94 & 1.000\\
\hline
2018-08-01 & CO-05 & 0.06 & 0.027\\
\hline
2018-08-01 & CO-06 & 0.58 & 0.648\\
\hline
\end{tabular}
\end{table}

\end{frame}

\begin{frame}{Tidy Data}
\protect\hypertarget{tidy-data}{}

\begin{table}

\caption{\label{tab:unnamed-chunk-2}Tidy Combined (18,400 observations of 4 variables)}
\centering
\begin{tabular}[t]{l|l|l|r}
\hline
Date & Race & Predictive Method & Probability\\
\hline
2018-08-01 & AZ-S1 & market & 0.660\\
\hline
2018-08-01 & AZ-S1 & model & 0.738\\
\hline
2018-08-01 & CA-12 & market & 0.910\\
\hline
2018-08-01 & CA-12 & model & 1.000\\
\hline
2018-08-01 & CA-22 & market & 0.300\\
\hline
2018-08-01 & CA-22 & model & 0.049\\
\hline
2018-08-01 & CA-25 & market & 0.610\\
\hline
2018-08-01 & CA-25 & model & 0.745\\
\hline
2018-08-01 & CA-39 & market & 0.610\\
\hline
2018-08-01 & CA-39 & model & 0.377\\
\hline
\end{tabular}
\end{table}

\end{frame}

\begin{frame}{Race Distributions}
\protect\hypertarget{race-distributions}{}

\includegraphics{../plots/plot_races_hist.png}

\end{frame}

\begin{frame}{Method Similarities}
\protect\hypertarget{method-similarities}{}

\includegraphics{../plots/plot_cart_points.png}

\end{frame}

\begin{frame}{Market Manipulation?}
\protect\hypertarget{market-manipulation}{}

\includegraphics{../plots/plot_nj_02.png}

\end{frame}

\begin{frame}{Method Accuracy}
\protect\hypertarget{method-accuracy}{}

\includegraphics{../plots/plot_prop_day.png}

\end{frame}

\begin{frame}{Method Accuracy}
\protect\hypertarget{method-accuracy-1}{}

\includegraphics{../plots/plot_prop_month.png}

\end{frame}

\end{document}
